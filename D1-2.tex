\documentclass[external]{20120615_deliverable_template_ukob}

\usepackage{xspace}
\usepackage{amsmath,amsthm,amssymb,url,color}
\theoremstyle{definition}
\newtheorem{example}{Example}
\usepackage{xcolor}
\usepackage{color, colortbl}
\usepackage{subfigure}
%\usepackage{showkeys}

\newcommand{\todo}[2]{\textcolor{magenta}{#1: #2}}

\LGtitle{\LiveGovThirtyTitleFonts \textbf{D1.2 - Mobile Sensor App with Mining Functionality} }
%\LGnumber{WP.NR} % e.g 6.7 no letters D or ID

\LGnumber{1.2}

\LGwp{WP1 - Reality Sensing and Mining}

\LGdissemination{PU - Public}

\LGcontractdate{Month 27, March 2014}

\LGactualdate{Month 26, February 2014}

\LGtask{T1.1, T1.2, T1.3}

\LGtype{Prototype}

%\LGnature{$<$ Report $|$ Prototype $|$ Demonstrator $|$ Other $>$}

%\LGapproved % activate only if approved
\LGfinal

\LGversion{12}
%\LGstaffmonths{Resources spent on the deliverable}

%\LGdistribution{WP leaders, PMB members, European Commission}

%\LGkeywords{}


%\LGabstract{Put abstract here (separate paragraphs using
%{\tt\char92 par}.)}
\LGabstract{ 

\todo{HH}{Add abstract}

%  This deliverable describes the mobile sensor mining component.
%  Along with this document we supply the source code of the
%  components, documentation (javadoc), and pre-compiled packages for
%  direct installation and testing on mobile devices.  

}

%\LGhistory{01}{2011-10-01}{First draft}{Yiannis Kompatsiaris}
%\LGhistory{02}{2011-10-02}{Modifications}{Sotiris Diplaris}
%\LGhistory{03}{2011-10-03}{Further Modifications \& Corrections}{Sotiris Diplaris}
%\LGhistory{04}{2011-10-05}{Further Modifications \& Corrections}{Sotiris Diplaris}


%------------------------ macro for change portrait to landscape
\usepackage{calc,graphicx,pdflscape,eso-pic} % needed for printing
                                             % headers and footers
\newlength\landscapewidth
\newlength\landscapeheight
\newcommand\landscapepagestyle{ % command which prepare page
                                % for landscape printing,
                                % change to landscape is
                                % achieved by pdflscape
\clearpage \thispagestyle{empty}
\setlength\landscapewidth{247mm}
\setlength\landscapeheight{161mm}
\AddToShipoutPicture*{\AtPageCenter{ % this make "layer" for
                                     % printing the header and footer
\rotatebox[origin=c]{90}{
\hspace*{-2em} % for adjusting position of
               % header and footer
\parbox{\landscapewidth}{\vskip-.57\landscapeheight
%\centerline{\WEKNOWIT \hfill \textbf{\Large \bf{IDx.x - V0} \normalsize }} %header text for Internal Deliverables
                                                                         % notation: IDx.x V{LGversion}
\centerline{\LiveGovLogo \hfill \textbf{\Large \bf{Dx.x - V0} \normalsize }} %header text for External Deliverables
                                                                        % notation: Dx.x V{LGversion}
\vspace{-5pt} \rule{\landscapewidth}{0.4pt} \\
\rule{0pt}{\landscapeheight} \\
%\rule{\landscapewidth}{0.4pt} \\
\centerline{Page \thepage\ }  % footer text
} } } } }

%--------------------------------------------------------------------------------------------------------------%
\usepackage{appendix}
\pretolerance=10000 % prevent overflow lines for $math$ elements

\begin{document}

% add a \LGaddhistory{version}{date}{reason}{revised by} for each new
% version
\begin{LGhistory}
\LGaddhistory{01}{2013-10-14}{Outline}{Heinrich Hartmann}
\LGaddhistory{02}{2014-01-06}{Added section on topic modeling}{Christoph Kling}
\LGaddhistory{03}{2014-01-07}{Revised Outline}{Heinrich Hartmann}

\end{LGhistory}


\newcommand{\LGaddauthorNoPhone}[3]{\hline  #1 &  #2 & %
   \parbox{3em}{E-mail:} \small #3 \\
}

% add a
% \LGaddauthor{Partner}{Name}{Telephone}{Fax}{Email} for each author
\begin{LGauthors}

\LGaddauthor{UKob}{Heinrich Hartmann}{+49 261 287 2759}%
{+49 261 287 100 2759}{\small hartmann@uni-koblenz.de}

\LGaddauthor{UKob}{Christoph Schaefer}{+49 261 287 2786 }%
{+49 261 287 100 2786}{\small chrisschaefer@uni-koblenz.de}

\LGaddauthor{UKob}{Christoph Kling}{+49 261 287 2702}%
{+49 261 287 100 2702}{\small ckling@uni-koblenz.de}

\LGaddauthorNoPhone{MTS}{Laura Niittyl\"a}%
{\small Laura.Niittyla@mattersoft.fi}

\end{LGauthors}



\begin{LGExecutiveSummary}
  \vspace{10pt} 

\todo{HH}{Add Summary}

\end{LGExecutiveSummary}


% add a \LGaddabbreviation{ABBR}{Explanation} for each abbreviation
\begin{LGAbbreviations}
%\LGaddabbreviation{LG}{Live+Gov}

\LGaddabbreviation{\textbf{API}}
{Application Programming Interface}

\LGaddabbreviation{\textbf{GPS}}
{Global Positioning System}

\LGaddabbreviation{\textbf{GSM}}
{Global System for Mobile Communications}

\LGaddabbreviation{\textbf{HTML}}
{HyperText Markup Language}

\LGaddabbreviation{\textbf{HTTP}}
{Hypertext Transfer Protocol}

\LGaddabbreviation{\textbf{JSON}}
{JavaScript Object Notation}

\LGaddabbreviation{\textbf{REST}}
{Representational State Transfer}

\LGaddabbreviation{\textbf{SVM}}
{Support Vector Machine}

\LGaddabbreviation{\textbf{URL}}
{Uniform Resource Locator}

\LGaddabbreviation{\textbf{UUID}}
{Universal Unique Device Identifier}

\LGaddabbreviation{\textbf{WP}}
{Work Package}

\LGaddabbreviation{\textbf{WIFI}}
{Wireless Fidelity (IEEE 802.11), WLAN}

\LGaddabbreviation{\textbf{WLAN}}
{Wireless Local Area Network}

\LGaddabbreviation{\textbf{XML}}
{Extensible Markup Language}

\LGaddabbreviation{~\\}
{~~~}

\end{LGAbbreviations}

% add a \LGaddterm{Term}{Explanation} for each entry of the glossary
%\begin{LGGlossary}
%\LGaddterm{Term}{Definition text here}
%\LGaddterm{Term2}{Definition text here}
%\end{LGGlossary}

\setcounter{tocdepth}{1}

% add this for a table of contents
\LGTOC

\newpage

\chapter{Introduction}
\label{chap:Introduction}
\todo{HH}{Write Introduction}



\clearpage

%--------------------------------------------------------------------
\chapter{Topic Models}
%--------------------------------------------------------------------

The BuitenBeter dataset consists of 13,811 geo-tagged issues reported to the office of public order
in the Netherlands between July and September 2012.  Each issue is assigned to one of 16 categories
such as "poor road conditions", "Dirt on the street" or "Graffiti".

Geographical co-occurrences of arbitrary categorical observations - such as public order issues of
the BuitenBeter dataset - can be used to discover latent underlying factors that explain
observations which co-occur more frequently than expected by chance.  One method for detecting these
latent factors is probabilistic topic modelling, where grouped observations are modelled as being
sampled from mixtures of latent topics, which are probability distributions over the set of distinct
observations.

For the BuitenBeter dataset, we e.g. might expect that areas where graffities are reported will see
many reports of the category "Dirt on the street" since both categories are related to urban
environments. We model those environments as hidden "topics".
  
Existing approaches for geographical topic modelling adopt topic models such as latent Dirichlet
allocation \cite{Blei:2003:LDA:944919.944937} and extend the models by assigning distributions over
locations to topics, or by introducing latent geographical regions. In models which extend topics
for spatial distributions (such as two-dimensional normal distributions) \cite{conf/wsdm/Sizov10},
topics with a complex (i.e. non-Gaussian) spatial distribution cannot be detected. In models with
latent, Gaussian distributed regions \cite{conf/www/YinCHZH11}, documents within a complex shaped
topic area do not influence the topic distribution of distant documents within the same
area. Therefore, topics with a complex spatial distribution such as topics distributed along
coastlines, rivers or country borders are harder to detect by such methods.  More elaborate models
introduce artificial assumptions about the structure of geographical distributions, e.g. by
introducing hierarchical structures \cite{ahmed13} in advance.
%Additionally, some approaches \cite{conf/www/MeiLSZ06,conf/www/YinCHZH11} do not model document-specific
%topic distributions. 

We introduce a novel geographical topic model which captures dependencies between geographical
regions to support the detection of topics with complex, e.g. non-Gaus\-sian distributed spatial
structures \cite{CCK1}.
%The model is based on a hierarchical Dirichlet process
%extended to support multiple base distributions, which we name MDP. 
%Our method thus is called the MDP-based geographical topic model (MGTM). 
%We use the MDP to dynamically smooth topic distributions between groups of spatially
%adjacent documents.

Our model differs from existing approaches in several aspects:

We model locations and words separately, as the separation of spatial clusters and document
semantics allows us to define meaningful neighbour relations between spatially adjacent
clusters. Our topic model takes a set of geographical clusters as input.

We also expect that geographical clusters adjacent in space exhibit similar topic distributions:
Most geographical topics cannot be approximated by a simple spatial probability distribution such as
a Gaussian distribution and for these complex topic areas, coherent sets of multiple spatial
distributions are a reasonable approximation.  Therefore we detect and smoothe the topic
distribution of these adjacent regions to increase the probability of detecting coherent topic
areas.

The detection of geographical clusters is straight-forward: We use a mixture of a fixed number of
Fisher distributions -- distributions on a three dimensional unit sphere similar to isotropic
Gaussian distributions on a plane. The parameters are fit using the approximation given by Banerjee
et al.~\cite{DBLP:journals/jmlr/BanerjeeDGS05} in an expectation-maximisation algorithm.  In our
model, each cluster is associated with a distribution over the set of topics, sampled from a
Dirichlet process (DP) with a common base measure which is itself drawn from a DP with Dirichlet
distributed multinomial distributions over the set of categories as base measure.  For the smoothing
of topic distributions of adjacent clusters, we first define the adjacency relation using the
Delaunay triangulation \cite{journals/csur/Aurenhammer91} of the cluster centroids.  Each public
order issue report draws an own topic distribution from a DP with a weighted mixture of the topic
distribution of its own geographical cluster and its neighbour clusters as base measure.  Finally,
each observed report category is drawn from its document-specific topic distribution.  For a
detailed description of the model and its parameters, we refer to our paper \cite{CCK1}.

To train our model on the BuitenBeter dataset, we use 300 clusters and the initial settings for the
parameters $\beta = 0.5$, $\gamma = 1.0$, $\alpha_0 = 1.0$, $A = 99999$, $\delta = 1.0$. By setting
$A$ to such a high value, we give up document-specific topic distributions and sample the categories
directly from a mixture of region-specific topic distributions. All other parameters are set to the
values used in the paper \cite{CCK1}, except for the hyperparameters for $A$, which is fixed.

For creating a comprehensible report, the detected topics can be characterised by their most-likely
categories:

Topic 1: \textit{Dirt on the street}, \ \textit{Other}, \ \textit{Graffities\\}
Topic 2: \textit{Damaged street light}, \ \textit{Bad road}\\
Topic 3: \textit{Obstacle by trees}, \ \textit{Weed}, \ \textit{Loose paving stones}, \ \textit{Bad road}, \ \textit{Idea/wish}\\
Topic 4: \textit{Other}, \ \textit{Weed}, \ \textit{Loose paving stones}, \ \textit{Bad road}, \ \textit{Damaged street light}, \ \textit{Idea/wish}


%for reduzing the size, set the width to 0.22\textwid
\begin{figure}[]
\centering
\subfigure[Topic 1]{
  \includegraphics[width=0.3\textwidth,natwidth=556.19,natheight=432]{img/map/1.png}
  \label{plot:act20}
  \setcounter{subfigure}{1}
} 
\subfigure[Topic 2]{
  \includegraphics[width=0.3\textwidth,natwidth=556.19,natheight=432]{img/map/2.png}
  \label{plot:act20}
  \setcounter{subfigure}{2}
}

\subfigure[Topic 3]{
  \includegraphics[width=0.3\textwidth,natwidth=556.19,natheight=432]{img/map/3.png}
  \label{plot:act20}
  \setcounter{subfigure}{3}
}
\subfigure[Topic 4]{
  \includegraphics[width=0.3\textwidth,natwidth=556.19,natheight=432]{img/map/4.png}
  \label{plot:act20}
  \setcounter{subfigure}{4}
} 

\caption{
  Document positions of reports with above-average probabilities for Topic 1-4 on the map of the Netherlands.
}
\label{fig:netmaps}
\end{figure}

The position of documents with an above-average probability for each topic are shown in \vbox{Figure
  \ref{fig:netmaps}}.  We see that, as expected, Topic 1 is only observed in the area of larger
cities, whilst Topic 4 is present across the whole country.  Topic 2 occurs mostly within city
centres.  Topic 3 is observed in the area of Eindhoven for which different categories were available
in the BuitenBeter application.

\bibliography{D1-2}

\end{document}
